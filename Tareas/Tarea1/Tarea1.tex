\documentclass{article}
\usepackage{amsfonts} 
\usepackage{hyperref}
\usepackage{listings}
\usepackage{color}

\definecolor{dkgreen}{rgb}{0,0.6,0}
\definecolor{gray}{rgb}{0.5,0.5,0.5}
\definecolor{mauve}{rgb}{0.58,0,0.82}

\lstset{frame=tb,
  language=Python,
  aboveskip=3mm,
  belowskip=3mm,
  showstringspaces=false,
  columns=flexible,
  basicstyle={\small\ttfamily},
  numbers=none,
  numberstyle=\tiny\color{gray},
  keywordstyle=\color{blue},
  commentstyle=\color{dkgreen},
  stringstyle=\color{mauve},
  breaklines=true,
  breakatwhitespace=true,
  tabsize=3
}
\title{Tarea #1: Introducci\'{o}n}
\begin{document}
\maketitle
El objetivo de esta tarea es entender a mayor profundidad algunos de los
componentes utilizados para construir blockchains.
\\\\
Esta tarea intenta fomentar el trabajo de equipo y el intercambio de 
conocimientos entre personas de distintas facultades. Por eso mismo, 
por favor trabajen el deber juntos durante el periodo de clase 
o en cualquier otro momento de la semana. Solo se entregar una hoja por
clase.
\\\\
Este deber requiere un poco de programaci\'{o}n, la cual sera hecha por los
alumnos de ciencias de la computaci\'{o}n. Sin embargo, el objetivo es que todos
los estudiantes se familarizen con estos conceptos ya que seran utilizados a lo
largo del curso. Por eso pido que todos trabajen juntos y tengan discusiones
para que todos puedan entender los conceptos. La tarea es corta y el codigo
necesario es simple para que sea facil que el grupo lleve un ritmo harmonioso.
\\\\
Este deber sera entregado como un programa de python llamado \emph{tarea1.py} el cual
debe ser colocado en la carpeta ``Tareas/Tarea1/"" del repositorio de github de esta
clase: \url{https://github.com/netogallo/megaproyecto2017-2018}.

\section{Preparaci\'{o}n}
\begin{enumerate}
\item{Los estudiantes de ciencias de la computaci\'{o}n 
    deben crear una cuenta en github.com y enviarme 
    su usuario asi los agrego al repositorio de esta 
    clase que se encuentra en: \url{https://github.com/netogallo/megaproyecto2017-2018}}
\item{Para esta tarea se necesitara instalar 
    \href{https://www.python.org/downloads/release/python-361/}{Python 3}.}
\item{Durante este curso se utilizara git para llevar control de versiones y Github para 
    publicar avances. Los estudiantes de ciencias de la computación, si 
    no estan familiarizados con git, por favor seguir un 
    \href{https://try.github.io/levels/1/challenges/1}{tutorial}.}
\end{enumerate}

\section{Hashes Criptograficos}

Un hash criptografico es una funci\'{o}n $h: \{0,1\}^* \rightarrow \{0,1\}^n$ que
recibe como entrada informaci\'{o}n representada en binario y produce un numero
binario de $n$ digitos como salida. La asignaci\'{o}n de valores $v \in \{0,1\}^n$
del rango de $h$ (tambien llamados hashes) a las entradas $e \in \{0,1\}^*$ del dominio de $h$
debe suceder de tal forma que simule una asiganci\'{o}n aleatoria. En otras palabras,
si uno solamente tiene un valor $v \in \{0,1\}$ resultante de aplicar el hash $h$ a alguna 
entrada desconocida $e \in \{0,1\}^*$, es dificil determinar cual fue la entrada $e$ que se utilizo.
\\\\
En este incisio utilizaremos la biblioteca \emph{hashlib} de python para familiarizarnos
con los hashes cryptograficos. Ahora abra una terminal de python y ejecute el siguiente codigo.

\begin{lstlisting}
import hashlib
hashlib.sha256(b"Hola mundo").hexdigest()
\end{lstlisting}
\\
Esto imprime la representaci\'{o}n hexadecimal del hash criptografico \emph{sha-256} 
aplicado a la representacion binaria de ``Hola mundo". Este es el hash utilizado por
los Bitcoins.
\\\\
Para esta tarea utlizaremos una version simplificada del hash \emph{sha-256} llamado \emph{sha-32}. Esta
funci\'{o}n aplica el \emph{sha-256} a la entrada y toma los primerso 8 caracteres de la representaci\'{o}
hexadecimal del resultado y retorna eso como resultado. Por ejemplo:
\begin{lstlisting}
sha32(b"Hola mundo") #produce: "ca8f60b2"
\end{lstlisting} 
\\\\
Por favor crear una implementaci\'{o}n de la funci\'{o}n \emph{sha-32} y colocarla en
\emph{tarea1.py}. Se permite utilizar la funci\'{o}n \emph{hashlib.sha256} de python en su
implementaci\'{o}n.
\section{Nonce}
Poco a poco iremos aprendiendo acerca de las partes que conforman un blockchain.
Tal como dice el nombre, un blockchain es una cadena de blocks. Empezaremos por
explorar las partes que forman un block. Como primer punto de division, vamos
a dividir el block en dos partes: El \emph{cuerpo} y el \emph{nonce}. Para los propositos de
esta hoja, el cuerpo simplemente es informaci\'{o}n que no puede modificarse,
mientras que el nonce es una parte del block que puede tener un valor arbitrario.
Para este deber, los blocks seran representados por el string ``Block:" + nonce
donde nonce es un string de 8 caracteres hexadecimales (0-9 y a-f) en minusculas.
Por ejemplo: ``Block:abcd1234".
\\\\
Ahora proceda a calcular el valor \emph{sha-32} para los siguientes valores:
\begin{lstlisting}
print(sha32(str.encode("Block:00000000")))
print(sha32(str.encode("Block:00000001")))
print(sha32(str.encode("Block:00000002")))
\end{lstlisting}
\\\\
Observe que tan solo con cambiar un caracter en la entrada, se produce
un hash completamente diferente.
\\\\
Para finalizar esta secci\'{o}n, defina una funci\'{o}n llamada \emph{block} en \emph{tarea1.py} que
dado un numero entero, produce el block correspondiente a ese numero. Por ejemplo:
\begin{lstlisting}
block(0) #produce: "Block:00000000"
block(1) #produce: "Block:00000001"
\end{lstlisting}
\section{Target}
Los blockchains tienen asociado a ellos una dificultad. Esta dificultad
indica la cantidad de recursos computacionales que se deben invertir para
producir un block nuevo. Esta dificultad esta dada en funci\'{o}n de un
valor $t \in \{0,1\}^n$ llamado \emph{target}, donde $n$ es el numero de bits
generados por el hash. En el caso de \emph{sha-32}, $n$ es 32 (8 digitos hexadecimales).
\\\\
Para que un block sea aceptado en un blockchain, el hash
criptografico de ese block debe ser menor al target. En nuestro ejemplo,
si el target para nuestro ejemplo es ``00009999" (representado hexadecimalmente),
el block ``Block:00000010" con hash \emph{sha-32} de ``fc6d4fc4" seria rechazado por el
blockchain ya que ``fc6d4fc4" $\not<$ ``00009999". En cambio, el block ``Block:0001bdb9"
con hash ``00004def" seria aceptado por el blockchain ya que ``00004def" $<$ ``00009999".
\\\\
Como siguiente ejercicio, definir una funcion \emph{findBlock} en \emph{tarea1.py}
que recibe como parametro un target, y retorna el primer block cuyo \emph{sha-32}
retorne un valor menor a ese target. Por ejemplo:
\begin{lstlisting}
findBlock("000fffff") #produce: Block:0000023f
findBlock("0000ffff") #produce: Block:0001bdb9
findBlock("00000fff") #despues de un rato produce: Block:0033bb52
\end{lstlisting}
\\\\
Como pueden observar, cada vez que se agrega un cero al target, es mucho m\'{a}s
trabajoso encontrar un block que sea aceptado por el blockchain. En los blockchains
reales, este target se cambia automaticamente en intervalos determinados (cada 1024 blocks para Bitcoin)
de tal manera que a todas las computadoras que pertenecen al network de dicho blockchain
les tome una cantidad fija de tiempo (10 minutos para Bitcoin) en encontrar un nonce
con el cual el hash criptografico del block es menor al target.

\end{document}